%$Id: log.tex,v 25.95 2006/08/26 01:26:53 al Exp $ -*- LaTeX -*-
% man commands log .
% Copyright (C) 2001 Albert Davis
% Author: Albert Davis <aldavis@ieee.org>
%
% This file is part of "Gnucap", the Gnu Circuit Analysis Package
%
% This program is free software; you can redistribute it and/or modify
% it under the terms of the GNU General Public License as published by
% the Free Software Foundation; either version 2, or (at your option)
% any later version.
%
% This program is distributed in the hope that it will be useful,
% but WITHOUT ANY WARRANTY; without even the implied warranty of
% MERCHANTABILITY or FITNESS FOR A PARTICULAR PURPOSE.  See the
% GNU General Public License for more details.
%
% You should have received a copy of the GNU General Public License
% along with this program; if not, write to the Free Software
% Foundation, Inc., 51 Franklin Street, Fifth Floor, Boston, MA
% 02110-1301, USA.
%------------------------------------------------------------------------
\section{{\tt LOG} command}
\index{log command}
\index{batch mode}
\index{files}
\index{disk files}
\index{command record}
\index{record of commands}
\index{i-o redirection}
\index{redirection: i-o}
%------------------------------------------------------------------------
\subsection{Syntax}
\begin{verse}
{\tt log} {\it file}\\
{\tt log >>} {\it file}\\
{\tt log}
\end{verse}
%------------------------------------------------------------------------
\subsection{Purpose}

Saves a copy of your keyboard entries in a file.
%------------------------------------------------------------------------
\subsection{Comments}

The `{\tt >>}' option appends to an existing file, if it exists, otherwise it
creates one.

Files can be nested.  If you open one while another is already open, both
will contain all the information.

A bare {\tt log} closes the file.  Because of this, the last line of this
file is always {\tt log}.  Ordinarily, this will not be of any consequence,
but if a log file is open when you use this file as command input, this will
close it.  If more than one {\tt log} file is open, they will be closed in
the reverse of the order in which they were opened, maintaining nesting.

See also: `{\tt >}' and `{\tt <}' commands.
%------------------------------------------------------------------------
\subsection{Bugs}

The file is an exact copy of what you type, so it is suitable for {\tt
gnucap <file} from the shell.  It is NOT suitable for the {\tt <}
command in gnucap or the Spice-like mode {\tt gnucap file} without
{\tt <}.
%------------------------------------------------------------------------
\subsection{Examples}

\begin{description}

\item[{\tt log today}] Save the commands in a file {\tt today} in the
current directory.  If {\tt today} already exists, the old one is gone.

\item[{\tt log >> doit}] Save the commands in a file {\tt doit}.  If
{\tt doit} already exists, it is kept, and the new data is added to the
end.

\item[{\tt log runit.bat}] Use the file {\tt runit.bat}.

\item[{\tt log}] Close the file.  Stop saving.

\end{description}
%------------------------------------------------------------------------
%------------------------------------------------------------------------
