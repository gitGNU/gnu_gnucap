%$Id: plot.tex,v 25.95 2006/08/26 01:26:53 al Exp $ -*- LaTeX -*-
% man commands plot .
% Copyright (C) 2001 Albert Davis
% Author: Albert Davis <aldavis@ieee.org>
%
% This file is part of "Gnucap", the Gnu Circuit Analysis Package
%
% This program is free software; you can redistribute it and/or modify
% it under the terms of the GNU General Public License as published by
% the Free Software Foundation; either version 2, or (at your option)
% any later version.
%
% This program is distributed in the hope that it will be useful,
% but WITHOUT ANY WARRANTY; without even the implied warranty of
% MERCHANTABILITY or FITNESS FOR A PARTICULAR PURPOSE.  See the
% GNU General Public License for more details.
%
% You should have received a copy of the GNU General Public License
% along with this program; if not, write to the Free Software
% Foundation, Inc., 51 Franklin Street, Fifth Floor, Boston, MA
% 02110-1301, USA.
%------------------------------------------------------------------------
\section{{\tt PLOT} command}
\index{plot command}
\index{graphics: set-up}
\index{plotting: set-up}
%------------------------------------------------------------------------
\subsection{Syntax}
\begin{verse}
{\tt plot}\\
{\tt plot} {\it mode points ...} ...\\
{\tt plot} {\it mode + points ...} ...\\
{\tt plot} {\it mode - points ...} ...\\
{\tt plot} {\it mode} CLEAR
\end{verse}
%------------------------------------------------------------------------
\subsection{Purpose}

Select points in the circuit for graphic output.  Select graphic output.
%------------------------------------------------------------------------
\subsection{Status}

The plotting leaves something to be desired.  Only two signals can be
plotted at a time.  The output file is corrupt when plotting is on.
%------------------------------------------------------------------------
\subsection{Comments}

The `{\tt plot}' command selects where to look at the circuit, or where to
hook the oscilloscope probe.

There are separate lists of probe points for each type of analysis.

To list the probe points, use the bare command `{\tt plot}'.

Syntax for each point is {\it parameter(node)(limits)}, {\it
parameter(componentlabel)(limits)}, or {\it parameter(index)(limits)}.
Some require a dummy index.

For more information on the data available see the {\tt print} command.

You must set the scaling.  If you do not, the default range is fixed
at -5 to 5.  Gnucap cannot auto-scale because it generates the plot
during simulation, so the necessary information is not available yet.
Spice can auto-scale only because it waits for the simulation to
complete before producing any output.

{\tt Plot} uses the same variables as {\tt print}.  See the print command 
for a list of what is available.

The options {\tt plot} and {\tt noplot} on any analysis command turn
plotting on and off a single run.  The {\tt plot} command turns 
plotting on and tabular output off.  The {\tt print} command turns plotting 
off and tabular output on.

You can add to or delete from an existing list by prefixing with
{\tt +} or {\tt -}.  {\tt plot ac + v(3)} adds v(3) to the existing
set of AC probes.  {\tt plot ac - q(c5)} removes q(c5) from the
list.  You can use the wildcard characters {\tt *} and {\tt ?} when
deleting.

Plotting is limited to 2 items.
%------------------------------------------------------------------------
\subsection{Examples}

\begin{description}

\item[{\tt plot ac vm(12)(0,5) vm(13)(-5,5)}] The magnitude of the voltage
at node 12 with a range of 0 to 5, and node 13 with a range of -5 to 5 for
AC analysis.

\item[{\tt plot dc v(r26)}] The voltage across {\tt R26} for DC analysis.
Since there is no range, default values will be used.

\item[{\tt plot tran v(r83)(0,5) p(r83)(0,1u)}] Plot the voltage and power 
of {\tt R83} in the next transient analysis.  The voltage scale is 0 to 5.  
The power scale is 0 to 1 microwatt.

\item[{\tt plot}] List all the probes for all modes.

\item[{\tt plot dc}] Display the DC plot list.

\item[{\tt plot ac CLear}] Clear the AC list.

\end{description}
%------------------------------------------------------------------------
%------------------------------------------------------------------------
