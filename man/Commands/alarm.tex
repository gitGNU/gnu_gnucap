%$Id: alarm.tex,v 21.13 2002/03/25 05:37:03 al Exp $
% man commands alarm .
% Copyright (C) 2001 Albert Davis
% Author: Albert Davis <aldavis@ieee.org>
%
% This file is part of "Gnucap", the Gnu Circuit Analysis Package
%
% This program is free software; you can redistribute it and/or modify
% it under the terms of the GNU General Public License as published by
% the Free Software Foundation; either version 2, or (at your option)
% any later version.
%
% This program is distributed in the hope that it will be useful,
% but WITHOUT ANY WARRANTY; without even the implied warranty of
% MERCHANTABILITY or FITNESS FOR A PARTICULAR PURPOSE.  See the
% GNU General Public License for more details.
%
% You should have received a copy of the GNU General Public License
% along with this program; if not, write to the Free Software
% Foundation, Inc., 59 Temple Place - Suite 330, Boston, MA
% 02111-1307, USA.
%------------------------------------------------------------------------
\section{{\tt ALARM} command}
\index{alarm command}
\index{output selection}
\index{safe operating area}
%------------------------------------------------------------------------
\subsection{Syntax}
\begin{verse}
{\tt ALArm}\\
{\tt ALArm} {\it mode points ...} ...\\
{\tt ALArm} {\it mode + points ...} ...\\
{\tt ALArm} {\it mode - points ...} ...\\
{\tt ALArm} {\it mode} CLEAR
\end{verse}
%------------------------------------------------------------------------
\subsection{Purpose}

Select points in the circuit to check against user defined limits.
%------------------------------------------------------------------------
\subsection{Comments}

The `{\tt alarm}' command selects points in the circuit to check
against limits.  There is no output unless the limits are exceeded.
If the limits are exceeded a the value is printed.

There are separate lists of probe points for each type of analysis.

To list the points, use the bare command `{\tt alarm}'.

Syntax for each point is {\it parameter(node)(limits)}, {\it
parameter(componentlabel)(limits)}, or {\it
parameter(index)(limits)}. Some require a dummy index.

For more information on the data available see the {\tt print}
command.

You can add to or delete from an existing list by prefixing with
{\tt +} or {\tt -}.  {\tt alarm ac + v(3)} adds v(3) to the existing
set of AC probes.  {\tt alarm ac - q(c5)} removes q(c5) from the
list.  You can use the wildcard characters {\tt *} and {\tt ?} when
deleting.
%------------------------------------------------------------------------
\subsection{Examples}

\begin{description}

\item[{\tt alarm ac vm(12)(0,5) vm(13)(-5,5)}] Check magnitude of
the voltage at node 12 against a range of 0 to 5, and node 13
against a range of -5 to 5 for AC analysis.  Print a warning when
the limits are exceeded.

\item[{\tt alarm op id(m*)(-100n,100n)}] Check current in all mosfets.
In op analysis, print a warning for any that are outside the range of
-100 to +100 nanoamps.  The range goes both positive and negative so
it is valid for both N and P channel fets.

\item[{\tt alarm tran v(r83)(0,5) p(r83)(0,1u)}] Check the voltage
and power of {\tt R83} in the next transient analysis.  The voltage
range is 0 to 5.  The power range is 0 to 1 microwatt.  Print a
warning when the range is exceeded.

\item[{\tt alarm}] List all the probes for all modes.

\item[{\tt alarm dc}] Display the DC alarm list.

\item[{\tt alarm ac CLear}] Clear the AC list.

\end{description}
%------------------------------------------------------------------------
%------------------------------------------------------------------------
