%$Id: get.tex,v 21.13 2002/03/25 05:37:03 al Exp $
% man commands get .
% Copyright (C) 2001 Albert Davis
% Author: Albert Davis <aldavis@ieee.org>
%
% This file is part of "Gnucap", the Gnu Circuit Analysis Package
%
% This program is free software; you can redistribute it and/or modify
% it under the terms of the GNU General Public License as published by
% the Free Software Foundation; either version 2, or (at your option)
% any later version.
%
% This program is distributed in the hope that it will be useful,
% but WITHOUT ANY WARRANTY; without even the implied warranty of
% MERCHANTABILITY or FITNESS FOR A PARTICULAR PURPOSE.  See the
% GNU General Public License for more details.
%
% You should have received a copy of the GNU General Public License
% along with this program; if not, write to the Free Software
% Foundation, Inc., 59 Temple Place - Suite 330, Boston, MA
% 02111-1307, USA.
%------------------------------------------------------------------------
\section{{\tt GET} command}
\index{get command}
\index{load circuit from file}
\index{read circuit from file}
\index{retrieve circuit from file}
\index{file: get}
\index{file: read}
%------------------------------------------------------------------------
\subsection{Syntax}
\begin{verse}
{\tt GET} {\it filename}
\end{verse}
%------------------------------------------------------------------------
\subsection{Purpose}

Gets an existing circuit file, after clearing memory.
%------------------------------------------------------------------------
\subsection{Comments}

The first comment line of the file being read is taken as the `title'.  See
the {\tt title} command.

Comments in the circuit file are stored, unless they start with {\tt *+} in
which case they are thrown away.

`Dot cards' are set up, but not executed.  This means that variables and
options are changed, but simulation commands are not actually done.  As
an example, the {\tt options} command is actually performed, since it only
sets up variables.  The {\tt ac} card is not performed, but its parameters
are stored, so that a plain {\tt ac} command will perform the analysis
specified in the file.

Any circuit already in memory will be erased before loading the new circuit.
%------------------------------------------------------------------------
\subsection{Examples}

\begin{description}

\item[{\tt get amp.ckt}] Get the circuit file {\tt amp.ckt} from
the current directory.

\item[{\tt get /usr/foo/ckt/amp.ckt}] Get the file {\tt amp.ckt}
from the {\tt /usr/foo/ckt} directory.

\item[{\tt get npn.mod}] Get the file {\tt npn.mod}.
\end{description}
%------------------------------------------------------------------------
%------------------------------------------------------------------------
