%$Id: title.tex,v 25.95 2006/08/26 01:26:53 al Exp $ -*- LaTeX -*-
% man commands title
% Copyright (C) 2001 Albert Davis
% Author: Albert Davis <aldavis@ieee.org>
%
% This file is part of "Gnucap", the Gnu Circuit Analysis Package
%
% This program is free software; you can redistribute it and/or modify
% it under the terms of the GNU General Public License as published by
% the Free Software Foundation; either version 2, or (at your option)
% any later version.
%
% This program is distributed in the hope that it will be useful,
% but WITHOUT ANY WARRANTY; without even the implied warranty of
% MERCHANTABILITY or FITNESS FOR A PARTICULAR PURPOSE.  See the
% GNU General Public License for more details.
%
% You should have received a copy of the GNU General Public License
% along with this program; if not, write to the Free Software
% Foundation, Inc., 51 Franklin Street, Fifth Floor, Boston, MA
% 02110-1301, USA.
%------------------------------------------------------------------------
\section{{\tt TITLE} command}
\index{title command}
\index{headings}
%------------------------------------------------------------------------
\subsection{Syntax}
\begin{verse}
{\tt title}\\
{\tt title} {\it a line of text}
\end{verse}
%------------------------------------------------------------------------
\subsection{Purpose}

View and create the heading line for printouts and files.
%------------------------------------------------------------------------
\subsection{Comments}

There is a header line at the beginning of every file, to help you identify
it in the future.  This command sets up what it says.  It also sets up a
heading for printouts and graphs.

When you use the `{\tt get}' command to bring in a new circuit, it replaces the
title with the one in the file.  The `{\tt title}' command lets you change it,
for the next time it is written out.
%------------------------------------------------------------------------
\subsection{Examples}

\begin{description}

\item[{\tt title This is a test.}] Sets the file heading to `{\tt This is a
test.}' In the future, all files written will have `{\tt This is a test.}'
as their first line.

\item[{\tt title}] Displays the file heading.  In this case, it prints `{\tt
This is a test.}'

\end{description}
%------------------------------------------------------------------------
%------------------------------------------------------------------------
