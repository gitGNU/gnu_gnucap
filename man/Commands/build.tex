%$Id: build.tex,v 25.95 2006/08/26 01:26:53 al Exp $ -*- LaTeX -*-
% man commands build .
% Copyright (C) 2001 Albert Davis
% Author: Albert Davis <aldavis@ieee.org>
%
% This file is part of "Gnucap", the Gnu Circuit Analysis Package
%
% This program is free software; you can redistribute it and/or modify
% it under the terms of the GNU General Public License as published by
% the Free Software Foundation; either version 2, or (at your option)
% any later version.
%
% This program is distributed in the hope that it will be useful,
% but WITHOUT ANY WARRANTY; without even the implied warranty of
% MERCHANTABILITY or FITNESS FOR A PARTICULAR PURPOSE.  See the
% GNU General Public License for more details.
%
% You should have received a copy of the GNU General Public License
% along with this program; if not, write to the Free Software
% Foundation, Inc., 51 Franklin Street, Fifth Floor, Boston, MA
% 02110-1301, USA.
%------------------------------------------------------------------------
\section{{\tt BUILD} command}
\index{build command}
\index{new circuit}
\index{creating new circuit}
\index{add to circuit}
%------------------------------------------------------------------------
\subsection{Syntax}
\begin{verse}
{\tt build} \{{\it line}\}
\end{verse}
%------------------------------------------------------------------------
\subsection{Purpose}

Builds a new circuit, or replaces lines in an existing one.
%------------------------------------------------------------------------
\subsection{Comments}

{\tt Build} Lets you enter the circuit from the keyboard.  The prompt changes 
to {\tt >} to show that the program is in the build mode.

At this point, type in the circuit components in standard (Spice type)
netlist format.

Component labels must be unique.  If not, the old one is modified according
to the new data, keeping old values where no new ones were specified.

Ordinarily, components are added to the end of the list.  To insert at a
particular place, specify the label to insert in front of.  Example: {\tt
Build R77} will cause new items to be added before {\tt R77}, instead of at
the end.

In either case, components being changed or replaced do not change their
location in the list.

If it is necessary to start over, {\tt delete all} or {\tt clear} will erase
the entire circuit in memory.

To exit this mode, enter a blank line.
%------------------------------------------------------------------------
\subsection{Examples}

\begin{description}

\item[{\tt build}] Build a circuit.  Add to the end of the list.  This will
add to the circuit without erasing anything.  It will continue until you
exit or memory fills up.

\item[{\tt b}] This is the same as the previous example.  Only the first
letter of the `Build' is necessary.

\item[{\tt build R33}]	Insert new items in front of R33.

\end{description}
%------------------------------------------------------------------------
%------------------------------------------------------------------------
