%$Id: 0summary.tex,v 21.13 2002/03/25 05:37:03 al Exp $
% man commands summary .
% Copyright (C) 2001 Albert Davis
% Author: Albert Davis <aldavis@ieee.org>
%
% This file is part of "Gnucap", the Gnu Circuit Analysis Package
%
% This program is free software; you can redistribute it and/or modify
% it under the terms of the GNU General Public License as published by
% the Free Software Foundation; either version 2, or (at your option)
% any later version.
%
% This program is distributed in the hope that it will be useful,
% but WITHOUT ANY WARRANTY; without even the implied warranty of
% MERCHANTABILITY or FITNESS FOR A PARTICULAR PURPOSE.  See the
% GNU General Public License for more details.
%
% You should have received a copy of the GNU General Public License
% along with this program; if not, write to the Free Software
% Foundation, Inc., 59 Temple Place - Suite 330, Boston, MA
% 02111-1307, USA.
%------------------------------------------------------------------------
\section{Command Summary}
\index{command descriptions}
\index{command summary}

\begin{description}

\item[{\tt *}] Comment line.
\index{comment line}

\item[{\tt !}] Pass a command to the operating system.

\item[{\tt <}] Batch mode.

\item[{\tt >}] Direct the ``standard output'' to a file.

\item[{\tt AC}] Performs a small signal AC (frequency domain) analysis.
Sweeps frequency.

\item[{\tt ALARM}] Select points in circuit check against limits.

\item[{\tt ALTER}] Perform analyses in queue.  Changes follow.  (Not
implemented.)

\item[{\tt BUILD}] Build a new circuit or change an existing one.

\item[{\tt CHDIR}] Change current directory.

\item[{\tt CLEAR}] Delete the entire circuit, titles, etc.

\item[{\tt DC}] Performs a nonlinear DC analysis, for determining transfer
characteristics.  Sweeps DC input or component values.

\item[{\tt DELETE}] Delete a part, or group of parts.

\item[{\tt DISTO}] SPICE command not implemented.

\item[{\tt EDIT}] Edit the circuit description using your editor.

\item[{\tt END}] Perform analyses in queue.  New circuit follows.
(Implemented incorrectly.)

\item[{\tt EXIT}] Exits the program.  (Same as quit.)

\item[{\tt FANOUT}] List by node number, the branches that connect to each
node.

\item[{\tt FAULT}] Temporarily change a component.

\item[{\tt FOURIER}] Transient analysis, with results in
frequency domain.  (Different from SPICE.)

\item[{\tt GENERATOR}] View and set the transient analysis function
generator.

\item[{\tt GET}] Get a circuit from a disk file.  Deletes old one first.

\item[{\tt IC}] Set transient analysis initial conditions.
(Not implemented.)

\item[{\tt INCLUDE}] Include a file from disk.  Add it the what is
already in memory.

\item[{\tt INSERT}] Insert a node number.  (Make a gap.)

\item[{\tt LIST}] List the circuit on the console.

\item[{\tt LOG}] Save a record of commands.

\item[{\tt MARK}] Mark this time point, so transient analysis will restart
here.

\item[{\tt MERGE}] Get a file from disk.  Add it the what is already in
memory.

\item[{\tt MODIFY}] Change a value, node, etc.  For very simple changes.

\item[{\tt NODESET}] Preset node voltages, to assist convergence.  (Not
implemented.)

\item[{\tt NOISE}] SPICE command not implemented.

\item[{\tt OP}] Performs a nonlinear DC analysis, for determining quiescent
operating conditions.  Sweeps temperature.

\item[{\tt OPTIONS}] View and set system options.  (Same as set.)

\item[{\tt PAUSE}] Wait for key hit, while in batch mode.

\item[{\tt PLOT}] Select points in circuit (and their range) to plot.

\item[{\tt PRINT}] Select which points in the circuit to print as table.

\item[{\tt QUIT}] Exits the program.  (Same as exit.)

\item[{\tt SAVE}] Save the circuit in a file.

\item[{\tt SENS}] SPICE command not implemented.

\item[{\tt STATUS}] Display resource usage, etc.

\item[{\tt SWEEP}] Sweep a component.  (Loop function.)

\item[{\tt TEMP}] SPICE command not implemented.

\item[{\tt TF}] SPICE command not implemented.

\item[{\tt TITLE}] View and create the heading line for printouts and files.

\item[{\tt TRANSIENT}] Performs a nonlinear transient (time domain)
analysis.  Sweeps time.

\item[{\tt UNFAULT}] Undo faults.

\item[{\tt UNMARK}] Undo mark.  Release transient start point.

\item[{\tt WIDTH}] Set output width.

\end{description}
%------------------------------------------------------------------------
%------------------------------------------------------------------------
