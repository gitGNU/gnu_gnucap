%$Id: edit.tex,v 25.95 2006/08/26 01:26:53 al Exp $ -*- LaTeX -*-
% man commands edit .
% Copyright (C) 2001 Albert Davis
% Author: Albert Davis <aldavis@ieee.org>
%
% This file is part of "Gnucap", the Gnu Circuit Analysis Package
%
% This program is free software; you can redistribute it and/or modify
% it under the terms of the GNU General Public License as published by
% the Free Software Foundation; either version 2, or (at your option)
% any later version.
%
% This program is distributed in the hope that it will be useful,
% but WITHOUT ANY WARRANTY; without even the implied warranty of
% MERCHANTABILITY or FITNESS FOR A PARTICULAR PURPOSE.  See the
% GNU General Public License for more details.
%
% You should have received a copy of the GNU General Public License
% along with this program; if not, write to the Free Software
% Foundation, Inc., 51 Franklin Street, Fifth Floor, Boston, MA
% 02110-1301, USA.
%------------------------------------------------------------------------
\section{{\tt EDIT} command}
\index{edit command}
%------------------------------------------------------------------------
\subsection{Syntax}
\begin{verse}
{\tt edit} \\
{\tt edit} {\it file}
\end{verse}
%------------------------------------------------------------------------
\subsection{Purpose}

Use your editor to change the circuit.
%------------------------------------------------------------------------
\subsection{Comments}

The {\tt edit} command runs your editor on a copy of the circuit in memory, 
then reloads it.

{\tt Edit} {\it file} runs your editor on the specified {\it file}.

If you are only changing a component value, the {\tt modify} command may be
easier to use.

The program uses the {\tt EDITOR} environment variable to find the editor
to use.  The command fails if there is no {\tt EDITOR} defined.
%------------------------------------------------------------------------
\subsection{Examples}

\begin{description}

\item[{\tt edit}] Brings up your editor on the circuit.

\item[{\tt edit foo}] Edits the file {\tt foo} in your current directory.

\end{description}
%------------------------------------------------------------------------
%------------------------------------------------------------------------
