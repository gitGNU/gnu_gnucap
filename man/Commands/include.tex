%$Id: include.tex,v 20.14 2001/10/19 06:21:44 al Exp $
% man commands include .
% Copyright (C) 2001 Albert Davis
% Author: Albert Davis <aldavis@ieee.org>
%
% This file is part of "GnuCap", the Gnu Circuit Analysis Package
%
% This program is free software; you can redistribute it and/or modify
% it under the terms of the GNU General Public License as published by
% the Free Software Foundation; either version 2, or (at your option)
% any later version.
%
% This program is distributed in the hope that it will be useful,
% but WITHOUT ANY WARRANTY; without even the implied warranty of
% MERCHANTABILITY or FITNESS FOR A PARTICULAR PURPOSE.  See the
% GNU General Public License for more details.
%
% You should have received a copy of the GNU General Public License
% along with this program; if not, write to the Free Software
% Foundation, Inc., 59 Temple Place - Suite 330, Boston, MA
% 02111-1307, USA.
%------------------------------------------------------------------------
\section{{\tt INCLUDE} command}
\index{merge command}
\index{include command}
\index{load circuit from file}
\index{read circuit from file}
\index{retrieve circuit from file}
\index{file: get}
\index{file: include}
\index{file: merge}
\index{file: read}
%------------------------------------------------------------------------
\subsection{Syntax}
\begin{verse}
{\tt INClude} {\it filename}
\end{verse}
%------------------------------------------------------------------------
\subsection{Purpose}

Gets an existing circuit or model file, adds it to what is already in memory.
%------------------------------------------------------------------------
\subsection{Comments}

The first comment line of the file being read is the new title, and replaces
the existing title.

Comments in the circuit file are stored, unless they start with {\tt *+} in
which case they are thrown away.

`Dot cards' are interpreted the same as they would have been had the
file been simply inserted in place.  This means they are used as
presets if this file is included from a ``get'', or run if it is
included from a ``$<$''.
%------------------------------------------------------------------------
\subsection{Examples}

\begin{description}

\item[{\tt include npn.mod}] Include the file {\tt npn.mod}.

\end{description}
%------------------------------------------------------------------------
%------------------------------------------------------------------------
