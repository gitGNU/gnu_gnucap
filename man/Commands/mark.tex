%$Id: mark.tex,v 20.14 2001/10/19 06:21:44 al Exp $
% man commands mark .
% Copyright (C) 2001 Albert Davis
% Author: Albert Davis <aldavis@ieee.org>
%
% This file is part of "GnuCap", the Gnu Circuit Analysis Package
%
% This program is free software; you can redistribute it and/or modify
% it under the terms of the GNU General Public License as published by
% the Free Software Foundation; either version 2, or (at your option)
% any later version.
%
% This program is distributed in the hope that it will be useful,
% but WITHOUT ANY WARRANTY; without even the implied warranty of
% MERCHANTABILITY or FITNESS FOR A PARTICULAR PURPOSE.  See the
% GNU General Public License for more details.
%
% You should have received a copy of the GNU General Public License
% along with this program; if not, write to the Free Software
% Foundation, Inc., 59 Temple Place - Suite 330, Boston, MA
% 02111-1307, USA.
%------------------------------------------------------------------------
\section{{\tt MARK} command}
\index{mark command}
\index{transient reruns}
%------------------------------------------------------------------------
\subsection{Syntax}
\begin{verse}
{\tt MArk}
\end{verse}
%------------------------------------------------------------------------
\subsection{Purpose}

Remember circuit voltages and currents.
%------------------------------------------------------------------------
\subsection{Comments}

After the {\tt mark} command, the {\tt transient} and {\tt fourier} analysis
will continue from the values that were kept by the {\tt mark} command,
instead of progressing every time.

This allows reruns from the same starting point, which may be at any time,
not necessarily 0.
%------------------------------------------------------------------------
\subsection{Examples}

\begin{description}
\item[{\tt transient 0 1 .01}] A transient analysis starting at zero,
running until 1 second, with step size .01 seconds.  After this run, the
clock is at 1 second.

\item[{\tt mark}] Remember the time, voltages, currents, etc.

\item[{\tt transient}] Another transient analysis.  It continues from 1
second, to 2 seconds.  (It spans 1 second, as before.)  This command was not
affected by the {\tt mark} command.

\item[{\tt transient}] This will do exactly the same as the last one.  From
1 second to 2 seconds.  If it were not for {\tt mark}, it would have started
from 2 seconds.

\item[{\tt transient 1.5 .001}] Try again with smaller steps.  Again, it
starts at 1 second.

\item[{\tt unmark}] Release the effect of {\tt mark}.

\item[{\tt transient}] Exactly the same as the last time, as if we didn't
{\tt unmark}.  (1 to 1.5 seconds.)

\item[{\tt transient}] This one continues from where the last one left off:
at 1.5 seconds.  From now on, time will move forward.

\end{description}
%------------------------------------------------------------------------
%------------------------------------------------------------------------
