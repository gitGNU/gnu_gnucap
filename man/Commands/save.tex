%$Id: save.tex,v 25.95 2006/08/26 01:26:53 al Exp $ -*- LaTeX -*-
% man commands save .
% Copyright (C) 2001 Albert Davis
% Author: Albert Davis <aldavis@ieee.org>
%
% This file is part of "Gnucap", the Gnu Circuit Analysis Package
%
% This program is free software; you can redistribute it and/or modify
% it under the terms of the GNU General Public License as published by
% the Free Software Foundation; either version 2, or (at your option)
% any later version.
%
% This program is distributed in the hope that it will be useful,
% but WITHOUT ANY WARRANTY; without even the implied warranty of
% MERCHANTABILITY or FITNESS FOR A PARTICULAR PURPOSE.  See the
% GNU General Public License for more details.
%
% You should have received a copy of the GNU General Public License
% along with this program; if not, write to the Free Software
% Foundation, Inc., 51 Franklin Street, Fifth Floor, Boston, MA
% 02110-1301, USA.
%------------------------------------------------------------------------
\section{{\tt SAVE} command}
\index{save command}
\index{store the circuit}
\index{keep the circuit}
\index{file: save}
%------------------------------------------------------------------------
\subsection{Syntax}
\begin{verse}
{\tt save} {\it filename} \{{\it options} ...\}
\end{verse}
%------------------------------------------------------------------------
\subsection{Purpose}

Saves the circuit on the disk.
%------------------------------------------------------------------------
\subsection{Comments}

The file is in an ASCII format, so the list may be used as part of a
report.  It is believed to be compatible with other simulators such as 
Berkeley Spice to the extent that the capabilities are the same.  
Compatibility with commercial Spice derivatives may be a problem because
they all have proprietary extensions and are incompatible with each other.

If the file name specified already exists, the old file is deleted and
replaced by a new file of the same name, after asking you for permission.

You can save a part of a circuit.  See the {\tt list} command for more details.
%------------------------------------------------------------------------
\subsection{Examples}

\begin{description}

\item[{\tt save works.ckt}] Save the circuit in the file {\tt
works.ckt}, in the current directory.

\item[{\tt save}] Save the circuit.  Since you did not specify a file name,
it will ask for one.

\item[{\tt save partof.ckt R*}] Save a partial circuit, just the
resistors, to the file {\tt partof.ckt}.  (See the {\tt List}
command.)

%%%\index{models: save}
%%%\item[{\tt save q2n2222.mod}] Save this as a model file {\tt q2n2222.mod},
%%%so it can be called as a macro-model later.  Note that model files must
%%%start with a letter, so plain 2n2222.mod would be impossible to call.

\item[{\tt save /client/sim/ckt/no33}] You can specify a path name.

\end{description}
%------------------------------------------------------------------------
%------------------------------------------------------------------------
