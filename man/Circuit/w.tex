%$Id: w.tex,v 21.13 2002/03/25 05:37:03 al Exp $
% man circuit w .
% Copyright (C) 2001 Albert Davis
% Author: Albert Davis <aldavis@ieee.org>
%
% This file is part of "Gnucap", the Gnu Circuit Analysis Package
%
% This program is free software; you can redistribute it and/or modify
% it under the terms of the GNU General Public License as published by
% the Free Software Foundation; either version 2, or (at your option)
% any later version.
%
% This program is distributed in the hope that it will be useful,
% but WITHOUT ANY WARRANTY; without even the implied warranty of
% MERCHANTABILITY or FITNESS FOR A PARTICULAR PURPOSE.  See the
% GNU General Public License for more details.
%
% You should have received a copy of the GNU General Public License
% along with this program; if not, write to the Free Software
% Foundation, Inc., 59 Temple Place - Suite 330, Boston, MA
% 02111-1307, USA.
%------------------------------------------------------------------------
\section{{\tt W}: Current Controlled Switch}
%------------------------------------------------------------------------
\subsection{Syntax}
\begin{verse}
{\tt W}{\it xxxxxxx n+ n-- ce mname} \{{\it ic}\}\\
{\tt .ISWitch} {\it label n+ n-- ce mname} \{{\it ic}\}
\end{verse}
%------------------------------------------------------------------------
\subsection{Purpose}

Current controlled switch.
%------------------------------------------------------------------------
\subsection{Comments}

{\it N+} and {\it n--} are the positive and negative element nodes,
respectively.  {\it Ce} is the name of an element through which
the controlling current flows.  {\it Mname} is the model name.  A
switch is a resistor between {\it n+} and {\it n--}.  The value of
the resistor is determined by the state of the switch.

The resistance between {\it n+} and {\it n--} will be {\it RON}
when the controlling current (through {\it ce}) is above {\it IT}
+ {\it IH}.  The resistance will be {\it ROFF} when the controlling
current is below {\it IT} - {\it IH}.  When the controlling current
is between {\it IT} - {\it IH} and {\it IT} + {\it IH}, the resistance
will retain its prior value.

You may specify {\tt ON} or {\tt OFF} to indicate the initial state
of the switch when the controlling current is in the hysteresis
region.

{\tt RON} and {\tt ROFF} must have finite positive values.

The controlling element can be any simple two terminal element.
Unlike SPICE, it does not need to be a voltage source.
%------------------------------------------------------------------------
\subsection{Model Parameters}

\begin{description}

\item[{\tt IT} = {\it x}] Threshold current.  (Default = 0.)

\item[{\tt IH} = {\it x}] Hysteresis current.  (Default = 0.)

\item[{\tt RON} = {\it x}] On resistance.  (Default = 1.)

\item[{\tt ROFF} = {\it x}] Off resistance.  (Default = 1e12)

\end{description}
%------------------------------------------------------------------------
%------------------------------------------------------------------------
