%$Id: l.tex,v 25.95 2006/08/26 01:26:53 al Exp $ -*- LaTeX -*-
% man circuit l .
% Copyright (C) 2001 Albert Davis
% Author: Albert Davis <aldavis@ieee.org>
%
% This file is part of "Gnucap", the Gnu Circuit Analysis Package
%
% This program is free software; you can redistribute it and/or modify
% it under the terms of the GNU General Public License as published by
% the Free Software Foundation; either version 2, or (at your option)
% any later version.
%
% This program is distributed in the hope that it will be useful,
% but WITHOUT ANY WARRANTY; without even the implied warranty of
% MERCHANTABILITY or FITNESS FOR A PARTICULAR PURPOSE.  See the
% GNU General Public License for more details.
%
% You should have received a copy of the GNU General Public License
% along with this program; if not, write to the Free Software
% Foundation, Inc., 51 Franklin Street, Fifth Floor, Boston, MA
% 02110-1301, USA.
%------------------------------------------------------------------------
\section{{\tt L}: Inductor}
%------------------------------------------------------------------------
\subsection{Syntax}
\subsubsection{Device}
\begin{verse}
{\tt L}{\it xxxxxxx n+ n-- value}\\
{\tt L}{\it xxxxxxx n+ n-- expression}\\
{\tt L}{\it xxxxxxx n+ n-- value}
	\{{\tt IC=}{\it initial-current}\}\\
{\tt .inductor} {\it label n+ n-- expression}
\end{verse}
\subsubsection{Model (optional)}
\begin{verse}
{\tt .model} {\it mname} {\tt TABLE} \{{\it args}\}
\end{verse}
%------------------------------------------------------------------------
\subsection{Purpose}

Inductor, or general flux storage element.
%------------------------------------------------------------------------
\subsection{Comments}

{\it N+} and {\it n--} are the positive and negative element nodes,
respectively.  {\it Value} is the inductance in Henries.

The (optional) initial condition is the initial (time = 0) value of
the inductor current (in Amperes).  Note that the initial conditions
(if any) apply only if the {\tt UIC} option is specified on the {\tt
transient} command.

You may specify the {\it value} in any of these forms:

\begin{enumerate}
  
\item
A simple value.  This is the inductance in Henrys.
  
\item
An expression, as described in the behavioral modeling chapter.  The
expression can specify the flux as a function of current, or the
inductance as a function of time.

\item
A {\it model}, as described in the behavioral modeling chapter.  The
{\tt table} model describes a table of flux linkages vs. current.

\end{enumerate}
%------------------------------------------------------------------------
\subsection{Probes}

The following probes (Transient, DC, and OP analysis) are available in
addition to those available for all basic elements.  See the {\tt
print} command for documentation.

\begin{description}

\item[{\tt DT}]
Time step.  The internal time step used for this device for numerical
integration.  It is not necessarily the same as the global time step.
  
\item[{\tt TIME}]
Time.  The time of the most recent calculation of this device.  It is
not necessarily the same as the global time.
  
\item[{\tt TIMEOLD}] 
The time of the previous calculation of this device.  It is not
necessarily the same as the global time.
  
\item[{\tt TIMEFUTURE}]
The latest recommended time for the next sample, as determined by this
device.  The actual time will probably be sooner than this.

\item[{\tt DTREQUIRED}]
The required approximate maximum time step, usually based on
truncation error estimation.
  
\item[{\tt FLUX}]
The flux linkages stored in this inductor, in Weber-turns.
  
\item[{\tt INDUCTANCE}]
The effective inductance of this device.  For a fixed inductor, it
be its value.  For a nonlinear inductor, it is the effective
inductance at this time, or $\partial \phi / \partial v$.
  
\item[{\tt DLDT}]
The time derivative of inductance.  For a linear inductor it will be
zero.
  
\item[{\tt DL}]
The change in inductance compared to the previous sample.  Its
primary use is in debugging models and numerical problems.  For a
linear inductor it will be zero.
  
\item[{\tt DFDT}]
The time derivative of flux.  Hopefully this is the same as voltage,
but it is calculated a different way and can be used as an accuracy
check.
  
\item[{\tt DFLUX}]
The change in flux linkages compared to the previous sample.  Its
primary use is in debugging models and numerical problems.

\end{description}
%------------------------------------------------------------------------
%------------------------------------------------------------------------
