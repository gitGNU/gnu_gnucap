%$Id: x.tex,v 21.13 2002/03/25 05:37:03 al Exp $
% man circuit x .
% Copyright (C) 2001 Albert Davis
% Author: Albert Davis <aldavis@ieee.org>
%
% This file is part of "Gnucap", the Gnu Circuit Analysis Package
%
% This program is free software; you can redistribute it and/or modify
% it under the terms of the GNU General Public License as published by
% the Free Software Foundation; either version 2, or (at your option)
% any later version.
%
% This program is distributed in the hope that it will be useful,
% but WITHOUT ANY WARRANTY; without even the implied warranty of
% MERCHANTABILITY or FITNESS FOR A PARTICULAR PURPOSE.  See the
% GNU General Public License for more details.
%
% You should have received a copy of the GNU General Public License
% along with this program; if not, write to the Free Software
% Foundation, Inc., 59 Temple Place - Suite 330, Boston, MA
% 02111-1307, USA.
%------------------------------------------------------------------------
\section{{\tt X}: Subcircuit Call}
%------------------------------------------------------------------------
\subsection{Syntax}
\begin{verse}
{\tt X}{\it xxxxxxx n1} \{{\it n2 n3 ...}\} {\it subname}
\end{verse}
%------------------------------------------------------------------------
\subsection{Purpose}

Subcircuit call
%------------------------------------------------------------------------
\subsection{Comments}

Subcircuits are used by specifying pseudo-elements beginning with {\tt X},
followed by the connection nodes.
%------------------------------------------------------------------------
\subsection{Probes}

\begin{description}

\item[{\tt V}{\it x}]  Port (terminal node) voltage.  {\it x} is
which port to probe.  1 is the first node in the "X" statement, 2
is the second, and so on.

\item[{\tt P}] Power.  The sum of the power probes for all the internal elements.

\item[{\tt PD}] Power dissipated.  The total power dissipated as heat.

\item[{\tt PS}] Power sourced.  The total power generated.

\end{description}

In this release, there are no probes available in AC analysis except for the
internal elements.  More parameters will be added.  Internal elements can be
probed by concatenating the internal part label with the subcircuit label.
R5.X7 is R5 inside X7.
%------------------------------------------------------------------------
%------------------------------------------------------------------------
