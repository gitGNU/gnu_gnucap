%$Id: tcap.tex,v 21.13 2002/03/25 05:37:03 al Exp $
% man circuit tcap .
% Copyright (C) 2001 Albert Davis
% Author: Albert Davis <aldavis@ieee.org>
%
% This file is part of "Gnucap", the Gnu Circuit Analysis Package
%
% This program is free software; you can redistribute it and/or modify
% it under the terms of the GNU General Public License as published by
% the Free Software Foundation; either version 2, or (at your option)
% any later version.
%
% This program is distributed in the hope that it will be useful,
% but WITHOUT ANY WARRANTY; without even the implied warranty of
% MERCHANTABILITY or FITNESS FOR A PARTICULAR PURPOSE.  See the
% GNU General Public License for more details.
%
% You should have received a copy of the GNU General Public License
% along with this program; if not, write to the Free Software
% Foundation, Inc., 59 Temple Place - Suite 330, Boston, MA
% 02111-1307, USA.
%------------------------------------------------------------------------
\section{Trans-capacitor}
%------------------------------------------------------------------------
\subsection{Syntax}
\begin{verse}
  {\tt .TCAPacitor} {\it label n+ n-- nc+ nc-- expression}\\
  {\tt .TCAPacitor} {\it label n+ n-- nc+ nc-- value}
        \{{\tt IV=}{\it initial-voltage}\}\\
  {\tt .TCAPacitor} {\it label n+ n-- model} \{{\tt L=}{\it length}\}
        \{{\tt W=}{\it width}\} \{{\tt IC=}{\it initial-voltage}\}
\end{verse}
%------------------------------------------------------------------------
\subsection{Purpose}

Trans-capacitor, or charge transfer device.
%------------------------------------------------------------------------
\subsection{Probes}

All probes that apply to ordinary capacitors also apply here.
%------------------------------------------------------------------------
\subsection{Comments}

{\it N+} and {\it n--} are the positive and negative element nodes,
respectively.  {\it Nc+} and {\it nc--} are the positive and negative
controlling nodes, respectively. 

This device places a charge between the output nodes that depends on
the voltage on its input nodes.  If you parallel the input with the
output, it becomes an ordinary capacitor.  While the use of this
device may appear straightforward, be careful.  It is easy to use it
in an unstable way.

All options, expressions, models, and probes that apply to ordinary
capacitors can also be used here.

It is used internally in some transistor models.
%------------------------------------------------------------------------
%------------------------------------------------------------------------
