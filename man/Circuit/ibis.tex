%$Id: ibis.tex,v 21.13 2002/03/25 05:37:03 al Exp $
% man circuit ibis .
% Copyright (C) 2001 Albert Davis
% Author: Albert Davis <aldavis@ieee.org>
%
% This file is part of "Gnucap", the Gnu Circuit Analysis Package
%
% This program is free software; you can redistribute it and/or modify
% it under the terms of the GNU General Public License as published by
% the Free Software Foundation; either version 2, or (at your option)
% any later version.
%
% This program is distributed in the hope that it will be useful,
% but WITHOUT ANY WARRANTY; without even the implied warranty of
% MERCHANTABILITY or FITNESS FOR A PARTICULAR PURPOSE.  See the
% GNU General Public License for more details.
%
% You should have received a copy of the GNU General Public License
% along with this program; if not, write to the Free Software
% Foundation, Inc., 59 Temple Place - Suite 330, Boston, MA
% 02111-1307, USA.
%------------------------------------------------------------------------
\section{Ibis models}
%------------------------------------------------------------------------
\subsection{Syntax}
\begin{verse}
{\tt .IBIs}{\it label nodes ... parameters ...}
\end{verse}
%------------------------------------------------------------------------
\subsection{Purpose}

IBIS models, used in signal integrity analysis.
%------------------------------------------------------------------------
\subsection{Comments}

The IBIS support in Gnucap is based on the draft standard for IBIS-X and
connectors, with emulation for traditional IBIS versions 4.0, 3.2,
2.1, and 1.1.

Two parameters are always required: {\tt File} and {\tt Select}.  {\tt
  File} simply specifies the file name.  {\tt Select} chooses what
parts of the file to simulate.

The node list and other parameters depend on what parts of the file
are selected.

Normally, you will {\tt select} one section of the file, which may
bring in others for completeness.  In general, you need to specify
both the class of object (in square brackets) and the name of the
section.  For example, you might select {\tt [Component]foo}, {\tt
  [Model]drive1}, or {\tt [Test\_Data]set1}.  The name must not have
any blanks, but you can substitute an underscore for a blank to match
IBIS keys that have blanks in them.

\subsubsection{[Component], [Board Description], or [Connector] sections}

These sections are designed to model a whole component, board, or
connector.  Because of this, they usually have more ``pins'' than you
want to use.  There can often be hundreds of them.  You must select
which ``pins'' you want to simulate.

To select the ``pins'', use the key {\tt Pins} followed by a list of
the pins you want, in parentheses.  The nodes correspond to these
``pins'' in order.  The simulator will use a reduced model, that
provides information only about these pins.

Example:

{\tt .ibis (2 4 5 7) file=hg2634.ibs select=[Component]hg26a pins=(a1,
  b2, c3, c5)}

\subsubsection{Sections defined by [Define] macros}

These sections include the traditional IBIS [Model], [Submodel],
[Test\_Load], and [Test\_Data] sections.  Most older ``spice''
simulators that support IBIS handle only this part.

The node list is defined by the IBIS-X macro.  In most cases, the
macro will be selected from the standard library.
%------------------------------------------------------------------------
\subsection{Parameters}

\begin{description}
  
\item[{\tt File} = {\it x}] File name.  This is the name of the IBIS
  file.  It usually has the extension {\tt .ibs} but may also have the
  extensions {\tt .ebd}, {\tt .pkg}, or {\tt .icn}.
  
\item[{\tt Select} = {\it x}] What to simulate within the model.  See
  the notes for its syntax.
  
\item[{\tt Pins} = ({\it x})] Pin selection list for component, board,
  or connector models.

\end{description}
%------------------------------------------------------------------------
\subsection{Probes}

\begin{description}

\item[{\tt V}] ``Pin'' voltage for a [Model].

\end{description}
%------------------------------------------------------------------------
%------------------------------------------------------------------------
