%$Id: generator.tex,v 20.14 2001/10/19 06:21:44 al Exp $
% man behavior generator .
% Copyright (C) 2001 Albert Davis
% Author: Albert Davis <aldavis@ieee.org>
%
% This file is part of "GnuCap", the Gnu Circuit Analysis Package
%
% This program is free software; you can redistribute it and/or modify
% it under the terms of the GNU General Public License as published by
% the Free Software Foundation; either version 2, or (at your option)
% any later version.
%
% This program is distributed in the hope that it will be useful,
% but WITHOUT ANY WARRANTY; without even the implied warranty of
% MERCHANTABILITY or FITNESS FOR A PARTICULAR PURPOSE.  See the
% GNU General Public License for more details.
%
% You should have received a copy of the GNU General Public License
% along with this program; if not, write to the Free Software
% Foundation, Inc., 59 Temple Place - Suite 330, Boston, MA
% 02111-1307, USA.
%------------------------------------------------------------------------
\section{{\tt GENERATOR}: Signal Generator time dependent value}
%------------------------------------------------------------------------
\subsection{Syntax}
\begin{verse}
{\tt GENERATOR} {\it scale}
\end{verse}
%------------------------------------------------------------------------
\subsection{Purpose}

The component ``value'' is dependent on a ``signal generator'',
manipulated by the ``generator'' command.
%------------------------------------------------------------------------
\subsection{Comments}

For transient analysis, the ``value'' is determined by a signal
generator, which is considered to be external to the circuit and part
of the test bench.  See the ``generator'' command for more
information.

For AC analysis, the value here is the amplitude.

Strictly, all of the functionality and more is available through the
Spice-like behavioral modeling functions, but this one provides a user 
interface closer to the function generator that an analog designer
would use on a real bench.  It is mainly used for interactive operation.

It also provides backward compatibility with predecessors to GnuCap,
which used a different netlist format.
%------------------------------------------------------------------------
%\subsection{Example} 
%------------------------------------------------------------------------
%------------------------------------------------------------------------
