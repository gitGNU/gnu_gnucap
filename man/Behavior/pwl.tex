%$Id: pwl.tex,v 21.13 2002/03/25 05:37:03 al Exp $
% man behavior pwl .
% Copyright (C) 2001 Albert Davis
% Author: Albert Davis <aldavis@ieee.org>
%
% This file is part of "Gnucap", the Gnu Circuit Analysis Package
%
% This program is free software; you can redistribute it and/or modify
% it under the terms of the GNU General Public License as published by
% the Free Software Foundation; either version 2, or (at your option)
% any later version.
%
% This program is distributed in the hope that it will be useful,
% but WITHOUT ANY WARRANTY; without even the implied warranty of
% MERCHANTABILITY or FITNESS FOR A PARTICULAR PURPOSE.  See the
% GNU General Public License for more details.
%
% You should have received a copy of the GNU General Public License
% along with this program; if not, write to the Free Software
% Foundation, Inc., 59 Temple Place - Suite 330, Boston, MA
% 02111-1307, USA.
%------------------------------------------------------------------------
\section{{\tt PWL}: Piecewise linear function}
%------------------------------------------------------------------------
\subsection{Syntax}
\begin{verse}
{\tt PWL} {\it x1,y1 x2,y2 ...}
\end{verse}
%------------------------------------------------------------------------
\subsection{Purpose}

Defines a piecewise linear transfer function or time dependent value.
%------------------------------------------------------------------------
\subsection{Comments}

This is similar to, but not exactly the same as, the Berkeley SPICE
PWL for fixed sources.

For capacitors, this function defines {\em charge} as a function of
voltage.  For inductors, it defines {\em flux} as a function of
current.

For fixed sources, it defines voltage or current as a function of
time.

The values of {\it x} must be in increasing order.

Outside the specified range, the behavior depends on the type of
element.  For fixed sources, the output (voltage or current) is
constant at the end value.  This is compatible with SPICE.  For other
types, the last segment is extended linearly.  If you want it to
flatten, specify an extra point so the slope of the last segment is
flat.
%------------------------------------------------------------------------
\subsection{Parameters}

There are no additional parameters, beyond those that apply to all.
%------------------------------------------------------------------------
\subsection{Example} 

\begin{description}

\item[{\tt C1 2 0 pwl -5,-5u 0,0 1,1u 4,2u 5,2u}] This ``capacitor''
stores 5 microcoulombs at -5 volts (negative, corresponding to the
negative voltage, as expected.  The charge varies linearly to 0 at 0
volts, acting like a 1 microfarad capacitor.  (C = dq/dv).  This
continues to 1 volt.  The 0,0 point could have been left out.  The
charge increases only to 2 microcoulombs at 4 volts, for an
incremental capacitance of 1u/3 or .3333 microfarads.  The same charge 
at 5 volts indicates that it saturates at 2 microcoulombs.  For
negative voltages, the slope continues.

\end{description}
%------------------------------------------------------------------------
%------------------------------------------------------------------------
